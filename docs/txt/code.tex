\documentclass[12pt]{report}
\usepackage[a4paper, total={16cm, 24cm}]{geometry}
\usepackage[greek]{babel}
\usepackage{listings}
\usepackage{xcolor}
\usepackage[acronym]{glossaries}

\definecolor{codegray}{rgb}{0.5,0.5,0.5}
\definecolor{backcolour}{rgb}{0.95, 0.95, 0.92}

\lstdefinestyle{style}{
    backgroundcolor=\color{backcolour},   
    basicstyle=\ttfamily\footnotesize,
    breakatwhitespace=false,         
    breaklines=true,                 
    keepspaces=true,                 
    showspaces=false,                
    showstringspaces=false,
    showtabs=false,                  
    tabsize=2
}
\lstset{style=style}

\begin{document}
Παρακάτω θα αναλυθούν οι συναρτήσεις που αποτελούν τον κορμό των λειτουργιών του προγράμματος και που βρίσκονται
στο αρχείο {\latintext{processData.py}}. Η επεξήγηση του κυρίως προγράμματος {\latintext{main.py}} θα γίνει μέσω αυτής της διαδικασίας.

Η πρώτη συνάρτηση ονομάζεται {\latintext{formatData}} και σκοπός της είναι να μετατρέψει τα δεδομένα
σε μορφή που θα επιτρέψει και θα διευκολύνει την επεξεργασία τους.

Τα δύο κύρια αρχεία δεδομένων που χρησιμοποιεί το πρόγραμμα είναι η χρονοσειρά του φορτίου του δικτύου και η 
χρονοσειρά της παραγώμενης ισχύος των ΦΒ, για την συγκεκριμένη περιοχή. 

Τα δεδομένα για το φορτίο του δικτύου λαμβάνονται έτοιμα για την περιοχή που θα επιλεχθεί, από το αρχείο στο οποίο
είναι αποθηκευμένα με όνομα {{\latintext{data/}}\textit{<περιοχή>}{\latintext{\_gridload.csv}}. 
Ο χρήστης στην αρχή του προγράμματος ζητείται να διαλέξει μία από τις περιοχές που αναγράφονται και ανάλογα με την
επιλογή του (1-5), επιλέγεται το κατάλληλο αρχείο από τον παραπάνω φάκελο.

{\latintext{
\begin{lstlisting}[language=Python] 
print("\nSelect examination area from the list below (1-5)")
place = int(input("[1] Chania\n[2] Rethymno\n"
                  + "[3] Heraklio\n[4] Ag.Nikolaos\n[5] Moires\n"))

while place > 5 or place < 1:
    print("\nInvalid answer. Please choose one of the below:")
    place = int(input("[1] Chania\n[2] Rethymno\n"
                      + "[3] Heraklio\n[4] Ag.Nikolaos\n[5] Moires\n"))

\end{lstlisting}
}}

Όσον αφορά τα δεδομένα για την παραγώμενη ισχύς από τα ΦΒ, αυτά κατεβαίνουν από την ιστοσελίδα του 
{\latintext{PV-GIS}} μέσω του {\latintext{API}} του. Το αίτημα για τα δεδομένα γίνεται 
μέσω της εντολής {\latintext{curl}} και τα αυτά αποθηκεύονται με τη σειρά τους στο αρχείο 
{\latintext{data/pv\_production.csv}}.
Για να γίνει το αίτημα αυτό στην ιστοσελίδα, θα πρέπει να δωθούν κάποιες παράμετροι, μέσα στις οποίες είναι και οι 
συντεταγμένες της περιοχής. Συνεπώς ανάλογα με την επιλογή του χρήστη, οι συντεταγμένες αποθηκεύονται σε δύο 
μεταβλητές, {\latintext{lat}} και {\latintext{lon}}. Αυτές αντλούνται από ένα {\latintext{dictionary}}, το οποίο
ορίζεται στην αρχή του προγράμματος για να είναι εφικτή η εύκολη τροποποίησή του ανάλογα με τις ανάγκες του χρήστη.
Για τον ίδιο λόγο ορίζονται στην αρχή του προγράμματος και οι υπόλοιπες μεταβλητές που χρειάζεται η ιστοσελίδα, 
αλλά και όλο το πρόγραμμα.

{\latintext{
\begin{lstlisting}[language=Python] 
places = {
    1: (35.512, 24.012),
    2: (35.364, 24.471),
    3: (35.343, 25.153),
    4: (35.185, 25.706),
    5: (35.050, 24.877)
}
\end{lstlisting}
}}

{\latintext{
\begin{lstlisting}[language=Python] 
lat = str(places[place][0])
lon = str(places[place][1])
\end{lstlisting}
}}

Η συνάρτηση {\latintext{formatData}} δέχεται ως ορίσματα τις δύο χρονοσειρές και τις μετατρέπει σε έναν πίνακα που η κάθε του σειρά
είναι μία μέρα του χρόνου και κάθε στήλη του είναι η αντίστοιχη ώρα της ημέρας, κατά την οποία πάρθηκε η μέτρηση. 
Στο τέλος, η συνάρτηση επιστρέφει δύο πίνακες οι οποίοι έχουν 365 γραμμές και 24 στήλες.
\end{document}

